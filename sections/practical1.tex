\subsection{Specifying models using \texttt{lme4} in \texttt{R}}
\label{sec:specifyingmodelsusinglme4inr}

This section and the next focus on \texttt{lme4}, an often used package to do multilevel modeling in \texttt{R} with maximum likelihood methods \citep{BatesEa2015}.

\paragraph{Varying intercepts}

The functions \texttt{lmer} and \texttt{glmer} extend the syntax of \texttt{lm} and \texttt{glm}.
The varying intercept model in (\ref{eq:glmm01}) is specified as follows in \texttt{R} (using informative variable names instead of Greek letters).

\vspace{0.5\baselineskip}

\begin{lstlisting}
glmer(formula = construction ~ given + (1 | lemma),
      family = binominal(link=logit), data = my.data)
\end{lstlisting}

The pipe operator \texttt{x1|x2} can be read as \textit{x1 varies by x2}.
The intercept is denoted by \texttt{1}, and hence \texttt{(1|lemma)} simply says that the intercept varies by lemma.

\paragraph{Varying intercepts and slopes}

The VIVS model in (\ref{eq:glmm05}) is specified as follows (only the formula).

\vspace{0.5\baselineskip}

\begin{lstlisting}
construction ~ given + (1 + given | lemma)
\end{lstlisting}

Before the pipe, the part of the model is repeated that should be modeled as varying by the grouping factor after the pipe.
If a varying slope is specified, a varying intercept is silently assumed.
The last formula can therefore be abbreviated to the following equivalent one.

\vspace{0.5\baselineskip}

\begin{lstlisting}
construction ~ given + (given | lemma)
\end{lstlisting}

In order to let \textit{only} the slope vary, the intercept has to be removed explicitly from the random part of the formula.

\vspace{0.5\baselineskip}

\begin{lstlisting}
construction ~ given + (given - 1 | lemma)
\end{lstlisting}

\paragraph{Multiple random effects}

When there is more than one random effect, several bracketed terms are added.
The following is the recommended specification for models like (\ref{eq:glmm07}), regardless of whether the effects are nested or crossed.

\vspace{0.5\baselineskip}

\begin{lstlisting}
construction ~ given + (1 | lemma) +
               (1 | semantics)
\end{lstlisting}

Sometimes the following notation is used for nested random effects, where \texttt{semantics} nests \texttt{lemma}.

\vspace{0.5\baselineskip}

\begin{lstlisting}
construction ~ given + (1 | semantics / lemma)
\end{lstlisting}

\texttt{lme4} expands this to the following underlying syntax, which shows more clearly that nesting is handled as a kind of interaction.

\vspace{0.5\baselineskip}

\begin{lstlisting}
construction ~ given + (1 | semantics) +
               (1 | semantics : lemma)
\end{lstlisting}

There is a random intercept for \texttt{semantics} and one for each combination of \texttt{semantics} and \texttt{lemma}.
While these notations are seemingly very explicit about the nesting structure, they are not necessary under normal circumstances.
If the grouping factor \texttt{lemma} is nested within \texttt{semantics} (see Table~\ref{tab:nested} for a similar situation), \texttt{lme4} automatically treats it as nested, and the results are exactly the same with all the three aforementioned notations.

However, the following specification is \textit{not} equivalent and leads to problematic results.

\vspace{0.5\baselineskip}

\begin{lstlisting}
construction ~ given + (1 | semantics) +
               (1 | lemma) +
               (1 | semantics : lemma)
\end{lstlisting}

This instructs \texttt{lme4} to estimate the variance of \texttt{lemma} not just restricted to the permutations of the levels of \texttt{lemma} and \texttt{semantics} (\ie \texttt{semantics:lemma}), but also outside of these specific permutations.
In the nested case, there are no occurrences outside of these permutations, however, and the variance for \texttt{lemma} alone will be estimated close (but not exactly equal) to $0$.
To compensate for the spurious estimate for \texttt{lemma}, the variance estimate for \texttt{semantics:lemma} will be shifted unpredictably.

\begin{table}
  \centering
  \begin{tabular}{lll}
    \toprule
    \textbf{Exemplar} & \textbf{Speaker}  & \textbf{Region}        \\
    \midrule
                    1 &           D      &         Tyneside       \\
                    2 &           D      &         Tyneside       \\
                    3 &           R      &         Tyneside       \\
                    4 &           R      &         Tyneside       \\
                    5 &           D      &         Greater London \\
                    6 &           D      &         Greater London \\
                    7 &           R      &         Greater London \\
                    8 &           R      &         Greater London \\
    \bottomrule
  \end{tabular}
  \caption{Illustration of nested factors, organised suboptimally}
  \label{tab:nestedwrong}
\end{table}

There is one situation where the explicit notation for nested factors is necessary.
This is when the data are stored in a suboptimal way.
Such a suboptimal version of Table~\ref{tab:nested} would look something like Table~\ref{tab:nestedwrong}.
Here, the speaker factor is encoded as the initial letter of the name only.
Hence, Daryl and Dale (coming from two different regions) cannot be distinguisehd from each other, and Riley and Reed cannot, either.
This leaves \texttt{lme4} no way of recognising that the data structure is nested, and the user has to explicitly provide that information.
It would, of course, be better \textit{not} to organise data that way.

\paragraph{Second-level predictors}

(\ref{eq:glmm08}) and (\ref{eq:glmm09}) have the following \texttt{lme4} syntax.

\vspace{0.5\baselineskip}

\begin{lstlisting}
construction ~ given + frequency + (1 | lemma)
\end{lstlisting}

If the data is organised as shown in Table~\ref{tab:multilevel} -- \ie, with the second-level regressor not having any variance within the levels of the grouping factor --, \texttt{lme4} will detect this and treat \texttt{frequency} as a second-level effect.
% This should be kept in mind when interpreting the results of the estimation.
However, second-level predictors for random slopes are more tricky to specify (see \citealt[280-282]{GelmanHill2006}).
Assuming that the effect of givenness varies with the lemma, which itself comes with a second-level model including frequency as a regressor, the specification looks as follows.

\vspace{0.5\baselineskip}

\begin{lstlisting}
construction ~ given + frequency +
               given : frequency +
	       (1 + given | lemma)
\end{lstlisting}

A second-level regressor on a varying slope is thus an interaction between a first-level and a second-level fixed effect.

